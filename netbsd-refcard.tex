% Copyright (c) 2008 The NetBSD Foundation, Inc.
% All rights reserved.
%
% This code is derived from software contributed to The NetBSD Foundation
% by Leonardo Taccari.
%
% Redistribution and use in source and binary forms, with or without
% modification, are permitted provided that the following conditions
% are met:
% 1. Redistributions of source code must retain the above copyright
%    notice, this list of conditions and the following disclaimer.
% 2. Redistributions in binary form must reproduce the above copyright
%    notice, this list of conditions and the following disclaimer in the
%    documentation and/or other materials provided with the distribution.
%
% THIS SOFTWARE IS PROVIDED BY THE NETBSD FOUNDATION, INC. AND CONTRIBUTORS
% ``AS IS'' AND ANY EXPRESS OR IMPLIED WARRANTIES, INCLUDING, BUT NOT LIMITED
% TO, THE IMPLIED WARRANTIES OF MERCHANTABILITY AND FITNESS FOR A PARTICULAR
% PURPOSE ARE DISCLAIMED.  IN NO EVENT SHALL THE FOUNDATION OR CONTRIBUTORS
% BE LIABLE FOR ANY DIRECT, INDIRECT, INCIDENTAL, SPECIAL, EXEMPLARY, OR
% CONSEQUENTIAL DAMAGES (INCLUDING, BUT NOT LIMITED TO, PROCUREMENT OF
% SUBSTITUTE GOODS OR SERVICES; LOSS OF USE, DATA, OR PROFITS; OR BUSINESS
% INTERRUPTION) HOWEVER CAUSED AND ON ANY THEORY OF LIABILITY, WHETHER IN
% CONTRACT, STRICT LIABILITY, OR TORT (INCLUDING NEGLIGENCE OR OTHERWISE)
% ARISING IN ANY WAY OUT OF THE USE OF THIS SOFTWARE, EVEN IF ADVISED OF THE
% POSSIBILITY OF SUCH DAMAGE.


% leaflet class (print/tex-leaflet)
\documentclass[notumble]{leaflet}

% Packages
\usepackage[utf8]{inputenc}
\usepackage[T1]{fontenc}
\usepackage[english]{babel}
\usepackage{graphicx}
\usepackage[obeyspaces]{url}

% Some commands
\newcommand{\man}[1]{\mbox{\texttt{#1}}}
\newcommand{\NetBSD}{\mbox{NetBSD}}

% Reference card informations
\title{NetBSD reference card}
\author{The NetBSD community}
\date{}

% I like the dotted lines
\CutLine*{1}
\CutLine*{2}
\CutLine*{3}
\CutLine*{4}
\CutLine*{5}
\CutLine*{6}

% Real reference card starts here
\begin{document}
\begin{figure}
\centering
% You can download this image from: http://www.netbsd.org/images/NetBSD-flag.png 
\includegraphics{imgs/NetBSD-flag.png}
\end{figure}

\maketitle
\thispagestyle{empty}

\section{Useful links}
\begin{description}
\item[Official site:] \url{http://www.netbsd.org/}
\item[Man pages:] \url{http://man.netbsd.org/}
\item[Pkgsrc.se:] \url{http://www.pkgsrc.se/}
\item[ISO:] \url{ftp://iso.netbsd.org/pub/NetBSD/iso/}
\item[Binary packages:] \url{ftp://ftp.NetBSD.org/pub/NetBSD/packages/current-packages/NetBSD/$(uname -p)/$(uname -r)/All}
\item[NetBSD Wiki:] \url{http://wiki.netbsd.se/}
\item[NetBSD Italia:] \url{http://netbsdit.altervista.it/}
\item[NetBSD Mexico Wiki:] \url{http://wiki.netbsd.org.mx/}
\end{description}


\section{Obtaining the sources}
\begin{itemize}
\item To obtain the sources we should set various useful variables for
\man{cvs(1)}:
\verb|CVS_RSH="ssh"| \\
\verb|CVSROOT="anoncvs@anoncvs.NetBSD.org:/cvsroot"| \\
\verb|export CVS_RSH CVSROOT|
\item Now we choose a directory for the sources:
\verb|cd /usr|
\item To obtain the sources (in this case the 4.0):
\verb|cvs co -r netbsd-4-0-RELEASE -P src|
\item To obtain pkgsrc (i.e. the 2008Q1 release):
\verb|cvs co -r pkgsrc-2008Q1 -P pkgsrc|
\end{itemize}

\section{Configuration and useful commands}
\subsection{Audio}
\begin{itemize}
\item To record audio (i.e. with a microphone) you can use
\man{audiorecord(1)}: \verb|audiorecord -p mic myrec.wav|
\item To play an audio file, recorded for example with
\man{audiorecord(1)} you might use \man{audioplay(1)}:
\verb|audioplay myrec.wav|
\item To turn up or down the volume you can use \man{mixerctl(1)},
the increment and decrement operators are supported:
\verb|mixerctl -w outputs.master++|
For a list of all possible variables:
\verb|mixerctl -a|
\end{itemize}

\subsection{Localization}
\begin{itemize}
\item To set your favourite language and/or charset you might use \mbox{export}
(or \mbox{setenv} on the \man{csh(1)}) and the valid variables, for example for
the Italian language and \mbox{UTF-8}:
\verb|export LANG="it_IT.UTF-8"| \\
\verb|export LC_ALL="it_IT.UTF-8"|
\item To print the current settings:
\verb|locale|
\item To print all available settings you might use:
\verb|locale -a|
\end{itemize}

\subsection{Managing users and groups}
\begin{itemize}
\item To add a user: \\
\verb|useradd -m -s /bin/ksh -G wheel user|
where \url{/bin/ksh} is a shell and \mbox{wheel} is a secondary group.
\item To change some user's options, i.e add him/her to a group:
\verb|usermod -G group user|
\item To delete a user: \verb|userdel user|
\end{itemize}

\subsection{Monitoring the system}
\begin{itemize}
\item To monitor various informations of your \NetBSD{} box you might use the
\man{systat(1)} tool (which uses \man{curses(3)}):
\verb|systat all| When you're on systat you can move on another displays
with \verb|:display|, i.e. \verb|:ps| or \verb|:net|, for a list of all the
available displays: \verb|:help|
\item With \man{envstat(4)} you might obtain the sensor(s) informations, for
example: \verb|envstat -i 2|
\item To obtain the pid of a process you might use \man{pgrep(1)}:
\verb|pgrep envstat|
\item To \emph{kill} a process without knowing its pid you might use: 
\verb|pkill gimp|
\item To change the priority of a process you can use \man{renice(8)}: 
\verb|renice +6 `pgrep vi`|
\end{itemize}

\subsection{rc.d}
\subsubsection{rc.conf}
The synopsis of \url{rc.conf} file is:
\verb\daemon|option=YES|NO|value\

\begin{description}
\item[auto\_ifconfig] automatically starts the network interfaces
(please see \mbox{ifconfig\_if}) \emph{(boolean)}
\item[defaultroute] set the default gateway, i.e.
\verb|defaultroute=192.168.1.1|
\item[dhclient] configure the network using \mbox{DHCP} (\mbox{auto\_ifconfig} and
\mbox{ifconfig\_if} not needed) \emph{(boolean)}
\item[hostname] set the hostname, i.e.
\verb|hostname=bianconiglio|
\item[ifconfig\_if] assigns an IP or other on that network interface, i.e.
\verb|ifconfig_rtk0="inet 192.168.1.4"|
\item[mixerctl] automatically starts the mixerctl configuration
\emph{(boolean)}
\item[postfix] starts \mbox{Postfix} \emph{(boolean)}
\item[sshd] starts the \mbox{OpenSSH} server \emph{(boolean)}
\end{description}

\subsubsection{Start and stop services and other operations}
\begin{itemize}
\item To start a service only one time, i.e. the \mbox{sshd} daemon: 
\verb|/etc/rc.d/sshd onestart|
\item To stop a service (for example \mbox{sshd}): 
\verb|/etc/rc.d/sshd stop|
\item To restart a service (for example \mbox{network}): 
\verb|/etc/rc.d/network restart|
\end{itemize}

\subsection{wscons}
\begin{itemize}
\item To set the keyboard's layout, i.e. the Dvorak's layout: 
\verb|wsconsctl -w encoding=us.dvorak|
\item To turn off the pc speaker: \verb|wsconsctl -w bell.volume=0|
\item To see all available variables: \verb|wsconsctl -a|,
and display's options: \verb|wsconsctl -ad|
\end{itemize}


\section{Pkgsrc}
\subsection{mk.conf}
The \url{mk.conf} synopsis is: \verb|option = values| or for concatenating
values:
\verb|option += values|. Put a \verb|-| as a prefix if you'd like to disable a
option.

\begin{description}
\item[ACCEPTABLE\_LICENSES] not free software or OSI licenses that we accept,
i.e.:
\verb|ACCEPTABLE_LICENSES += vim-license|
\item[CFLAGS] flag passed to the compiler, i.e
\verb|CFLAGS += -march=pentium3|
\item[FAILOVER\_FETCH] if the distfile's checksum doesn't match, download
again the distfile, \verb|FAILOVER_FETCH = yes|
\item[FETCH\_CMD] tool to use on the \emph{fetch} phase, i.e.:
\verb|FETCH_CMD = wget|
\item[FETCH\_RESUME\_ARGS] useful options for the \emph{resume}, for example
in the wget case: \verb|FETCH_RESUME_ARGS = -c|
\item[PKG\_DEFAULT\_OPTIONS] options used by all packages, for example:
\verb|PKG_DEFAULT_OPTIONS += mmx -nas|
\item[PKG\_OPTIONS.package] options used by a package, for example in the
\mbox{www/elinks} case: \verb|PKG_OPTIONS.elinks += nntp|
\item[X11\_TYPE] option useful to set the X11 type, for modular \mbox{X.org}:
\verb|X11_TYPE = modular|
\end{description}

\subsection{pkg part}
\begin{description}
\item[audit-packages] show all the vulnerabilities of installed packages
\item[bpm] pseudo-TUI interface useful to install binary packages
\item[download-vulnerability-list] download the vulnerability list
\item[pkg\_add] install a package, and all required dependencies, i.e:
\verb|pkg_add -v mplayer|
\item[pkg\_delete] uninstall a package, i.e. to delete a package and all its
useless dependencies: \verb|pkg_delete -R mplayer|
\item[pkg\_info] display informations about a package, for example:
\verb|pkg_info mplayer|
\item[pkg\_tarup] create from installed packages in our \mbox{LOCALBASE} a .tgz binary
package, for example: \verb|pkg_tarup -a mplayer|
\end{description}

\subsection{src part}
\subsubsection{Available targets}
\begin{description}
\item[clean] delete the current package's \mbox{WRKDIR}
\item[clean-depends] delete all the \mbox{WRKDIR} of the current
package's dependencies
\item[deinstall] delete a package, to delete the useless dependencies too:
\verb|make DEINSTALLDEPENDS=1 deinstall|
\item[fetch] fetch the \emph{distfile} from repositories
\item[fetch-list] like \emph{fetch} but fetch the dependencies too,
you might use it in this way: \verb\make fetch-list | sh\
\item[help] display a useful help about a pkgsrc topic, i.e.
\verb|make help topic=DEINSTALLDEPENDS|
\item[install] install the package and its dependencies into our \mbox{LOCALBASE},
i.e. \verb|cd www/elinks && make install|
\item[update] update a package and dependencies
\item[show-depends] display all the dependencies of a package
\item[show-var] show a pkgsrc's variable, for example: 
\verb|make show-var VARNAME=MAINTAINER|
\end{description}

\subsubsection{Various tools}
\begin{description}
\item[pkgclean] like \emph{clean}'s target but smarter
\item[pkgfind] tools useful to find packages, for example
\verb|pkgfind -C video|
\end{description}


\section{Other operations}
\subsection{\man{shutdown(8)}}
\begin{itemize}
\item To reboot a machine: \verb|shutdown -r now|
\item To shutdown a \NetBSD{} machine you might use \man{shutdown(8)}, i.e.
\verb|shutdown -p 5 "Hurry up!"|
\end{itemize}
\subsection{Kernel}
\begin{itemize}
\item To obtain information about selected options in the kernel you might use 
\man{config(1)}, i.e.: \verb|config -x /netbsd|
\item To obtain or set kernel states you might use
\man{sysctl(8)}, i.e.: \verb|sysctl -w security.curtain=1|
\end{itemize}
\subsection{Testing}
\begin{itemize}
\item To run the automated tests:
\verb\cd /usr/tests && atf-run | atf-report\
\item To configure the testing framework, edit the files in \url{/etc/atf}.
\end{itemize}


\end{document}
